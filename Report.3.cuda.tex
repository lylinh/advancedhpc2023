\documentclass[12pt]{article}
\renewcommand{\baselinestretch}{1.2}
\usepackage[utf8]{vietnam}
\usepackage[a4paper, total={6in, 8in}]{geometry}
\usepackage[english]{babel}
\usepackage{hyperref}
\usepackage{mathtools}
\usepackage{amssymb}
\usepackage{indentfirst}
\usepackage{graphicx}
\usepackage{minted}
\usepackage{ragged2e}
\usepackage{multirow}
\usepackage{subcaption}
\usepackage{xurl}
\usepackage{amsmath}
\usepackage{makecell}
\usepackage{algorithm}
\usepackage{algpseudocode}
\renewcommand\theadalign{bc}
\renewcommand\theadfont{\bfseries}
\renewcommand\theadgape{\Gape[4pt]}
\renewcommand\cellgape{\Gape[4pt]}
\usepackage{pbox}
\usepackage{graphicx}
\usepackage{diagbox}
\usepackage{listings}
\usepackage{xcolor}

\hypersetup{
    colorlinks=true,
    linkcolor=blue,
    citecolor=blue,
    urlcolor=blue,
}
\lstset{
    backgroundcolor=\color{white},   
    basicstyle=\footnotesize,       
    breakatwhitespace=false,         
    breaklines=true,                 
    captionpos=b,                    
    commentstyle=\color{red},    
    escapeinside={\%*}{*)},          
    extendedchars=true,              
    keepspaces=true,                 
    keywordstyle=\color{blue},       
    language=Python,                
    numbers=left,                    
    numbersep=5pt,                  
    showspaces=false,                
    showstringspaces=false,
    showtabs=false,                  
    tabsize=2
}
\begin{document}
\begin{center}
    \vspace*{1.8cm}
    \Large
    Labwork3\\
\end{center}


\begin{figure}[H]
\centering
    \includegraphics[height = 0.5\textheight, keepaspectratio]{figures/labwork3-original.jpg}
    \caption{The original image}
\end{figure}
\section{Program}
\subsection{Load an image from file}
\begin{lstlisting}[language=Python]
import numba
import numpy as np
import matplotlib.pyplot as plt
from numba import cuda
from numba import jit
from matplotlib.image import imread
import time
image_path = 'labwork3-original.jpg'
image = imread(image_path)
plt.subplot(131), plt.imshow(image), plt.title('Original')
\end{lstlisting}
\subsection{Flatten image into 1D array of RGB}
\begin{lstlisting}[language=Python]
image_height = image.shape[0]
image_weight = image.shape[1]
pixel_count = image_height * image_weight
blockSize = 64
gridSize = int(pixel_count/blockSize)
\end{lstlisting}
\subsection{Flatten image into 1D array of RGB}
\begin{lstlisting}[language=Python]
flattened_image = image.reshape(pixel_count, 3)
\end{lstlisting}
\subsection{Grayscale using CPU}
\begin{lstlisting}[language=Python]
def grayscale_cpu(image):
    image_grayscale = np.zeros((image_height, image_weight), dtype=np.uint8)
    for i in range(image_height):
        for j in range(image_weight):
            image_grayscale[i, j] = (int(image[i][j][0]) + int(image[i][j][1]) + int(image[i][j][2])) /3
    return image_grayscale
\end{lstlisting}
\subsection{Grayscale using GPU}
\begin{lstlisting}[language=Python]
@cuda.jit
def grayscale_gpu(src, dst):
    tidx = cuda.threadIdx.x + cuda.blockIdx.x * cuda.blockDim.x
    g = np.uint8((src[tidx, 0] + src[tidx, 1] + src[tidx, 2]) / 3)
    dst[tidx, 0] = dst[tidx, 1] = dst[tidx, 2] = g
\end{lstlisting}
\subsection{Implement on CPU and GPU}
\begin{lstlisting}[language=Python]
# implement with CPU
start_time_cpu = time.time()
grayscale_cpu_image = grayscale_cpu(image)
time_cpu = time.time() - start_time_cpu
print(f"Time processing on CPU: {time_cpu}s")
plt.imsave('grayscale_cpu.jpg', grayscale_cpu_image, cmap='gray')
plt.subplot(132), plt.imshow(grayscale_cpu_image, cmap='gray'), plt.title('Grayscale (CPU)')

#implement with GPU
blockSizes = [64,128,256,512,1024]
times_gpu = []

for blockSize in blockSizes:
  gridSize = int(pixel_count/blockSize)
  devSrc = cuda.to_device(flattened_image)
  devDst = cuda.device_array((pixel_count, 3), np.uint8)

  start_time_gpu = time.time()
  grayscale_gpu_image = grayscale_gpu[gridSize, blockSize](devSrc, devDst)
  time_gpu = time.time() - start_time_gpu
  times_gpu.append(time_gpu)

  hostDst = devDst.copy_to_host()
  grayscale_gpu_image = np.reshape(hostDst, (image_height,image_weight ,3))
  plt.imsave('grayscale_gpu.jpg', grayscale_gpu_image, cmap='gray')
  plt.subplot(133), plt.imshow(grayscale_gpu_image, cmap='gray'), plt.title('Grayscale (GPU)')

for b, t in zip(blockSizes, times_gpu):
    print(f"Block size: {b}, Time processing on GPU: {t}")
plt.show()
\end{lstlisting}
\section{Discuss the results}
The time processing on GPU with block sizes [64,128,256,512,1024] is faster than on CPU.
The number of block size increases , the time processing on GPU decreases.
\begin{lstlisting}[language=Python]
Time processing on CPU: 0.9385004043579102s
Block size: 64, Time processing on GPU: 0.10533261299133301
Block size: 128, Time processing on GPU: 0.00019812583923339844
Block size: 256, Time processing on GPU: 0.00016808509826660156
Block size: 512, Time processing on GPU: 0.00018453598022460938
Block size: 1024, Time processing on GPU: 0.0001323223114013672
\end{lstlisting}
\end{document}