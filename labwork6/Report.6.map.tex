\documentclass[12pt]{article}
\renewcommand{\baselinestretch}{1.2}
\usepackage[utf8]{vietnam}
\usepackage[a4paper, total={6in, 8in}]{geometry}
\usepackage[english]{babel}
\usepackage{hyperref}
\usepackage{mathtools}
\usepackage{amssymb}
\usepackage{indentfirst}
\usepackage{graphicx}
\usepackage{minted}
\usepackage{ragged2e}
\usepackage{multirow}
\usepackage{subcaption}
\usepackage{xurl}
\usepackage{amsmath}
\usepackage{makecell}
\usepackage{algorithm}
\usepackage{algpseudocode}
\renewcommand\theadalign{bc}
\renewcommand\theadfont{\bfseries}
\renewcommand\theadgape{\Gape[4pt]}
\renewcommand\cellgape{\Gape[4pt]}
\usepackage{pbox}
\usepackage{graphicx}
\usepackage{diagbox}
\usepackage{listings}
\usepackage{xcolor}

\hypersetup{
    colorlinks=true,
    linkcolor=blue,
    citecolor=blue,
    urlcolor=blue,
}
\lstset{
    backgroundcolor=\color{white},   
    basicstyle=\footnotesize,       
    breakatwhitespace=false,         
    breaklines=true,                 
    captionpos=b,                    
    commentstyle=\color{red},    
    escapeinside={\%*}{*)},          
    extendedchars=true,              
    keepspaces=true,                 
    keywordstyle=\color{blue},       
    language=Python,                
    numbers=left,                    
    numbersep=5pt,                  
    showspaces=false,                
    showstringspaces=false,
    showtabs=false,                  
    tabsize=2
}
\begin{document}
\begin{center}
    \vspace*{1.8cm}
    \Large
    Labwork6\\
\end{center}

\section{Labwork 6a}
\subsection{Program}
\begin{lstlisting}[language=Python]
import numba
import numpy as np
import matplotlib.pyplot as plt
from numba import cuda
from numba import jit
from matplotlib.image import imread
import time

# Load an image from a file
image_path = 'labwork6-original.jpg'
image = imread(image_path)
plt.imshow(image), plt.title('Original')
plt.show()

# Get the image dimensions
image_height = image.shape[0]
image_width = image.shape[1]

# Binarization Grayscale using GPU with 2D blocks
@cuda.jit
def binarization_gpu(src, dst, threshold):
    x = cuda.threadIdx.x + cuda.blockIdx.x * cuda.blockDim.x
    y = cuda.threadIdx.y + cuda.blockIdx.y * cuda.blockDim.y
    g = np.uint8((src[x, y, 0] + src[x, y, 1] + src[x, y, 2]) // 3)
    if g >= threshold:
      g = 255
    else:
      g = 0
    dst[x, y, 0] = dst[x, y, 1] = dst[x, y, 2] = g

# Implement with GPU
blockSizes = [(2,2),(4,4),(8, 8),(16, 16),(32,32),(32,22)]
times_gpu = []
threshold = 100
counter = 0 
sub = [133,134,135,136,137,138]
for block_size in blockSizes:
  devSrc = cuda.to_device(image)
  devDst = cuda.device_array((image_height, image_width, 3), np.uint8)

  grid_size_x = (image_height + block_size[0] - 1) // block_size[0]
  grid_size_y = (image_width + block_size[1] - 1) // block_size[1]
  start_time_gpu = time.time()
  binarization_gpu[(grid_size_x, grid_size_y), block_size](devSrc, devDst, threshold)
  time_gpu = time.time() - start_time_gpu
  times_gpu.append(time_gpu)
  print(f"Block size:({block_size[0]},{block_size[1]}), Time processing on GPU: {time_gpu}s")

  hostDst = devDst.copy_to_host()
  grayscale_gpu_image = np.array(hostDst)
  plt.imsave(f'labwork6a-binarization_gpu_({block_size[0]},{block_size[1]}).jpg', grayscale_gpu_image, cmap='gray')
  
  plt.title(block_size)
  plt.imshow(grayscale_gpu_image)
  plt.show()
\end{lstlisting}
\subsection{Result}
 \begin{lstlisting}[language=Python]
Block size:(2,2), Time processing on GPU: 0.12934446334838867s
Block size:(4,4), Time processing on GPU: 0.00103759765625s
Block size:(8,8), Time processing on GPU: 0.0003185272216796875s
Block size:(16,16), Time processing on GPU: 0.0004918575286865234s
Block size:(32,32), Time processing on GPU: 0.00030732154846191406s
\end{lstlisting}

\section{Labwork 6b}
\subsection{Program}
\begin{lstlisting}[language=Python]
import numba
import numpy as np
import matplotlib.pyplot as plt
from numba import cuda
from numba import jit
from matplotlib.image import imread
import time

# Load an image from a file
image_path = 'labwork6-original.jpg'
image = imread(image_path)
plt.subplot(131), plt.imshow(image), plt.title('Original')

# Get the image dimensions
image_height = image.shape[0]
image_width = image.shape[1]

# Brightness using GPU with 2D blocks
@cuda.jit
def brightness_gpu(src, dst, threshold, brightness):
    x = cuda.threadIdx.x + cuda.blockIdx.x * cuda.blockDim.x
    y = cuda.threadIdx.y + cuda.blockIdx.y * cuda.blockDim.y
    if brightness > 0:
      r = min(src[x, y, 0] + threshold, 255)
      g = min(src[x, y, 1] + threshold, 255)
      b = min(src[x, y, 2] + threshold, 255)
    else:
      r = max(src[x, y, 0] - threshold, 0)
      g = max(src[x, y, 1] - threshold, 0)
      b = max(src[x, y, 2] - threshold, 0)
    dst[x, y, 0] = r
    dst[x, y, 1] = g
    dst[x, y, 2] = b

# Implement with GPU
blockSizes = [(2,2),(4,4),(8, 8),(16, 16),(32,32),(32,22)]
times_gpu = []
threshold = 100


# Implement increasing brightness
brightness = 1
for block_size in blockSizes:
  devSrc = cuda.to_device(image)
  devDst = cuda.device_array((image_height, image_width, 3), np.uint8)

  grid_size_x = (image_height + block_size[0] - 1) // block_size[0]
  grid_size_y = (image_width + block_size[1] - 1) // block_size[1]
  start_time_gpu = time.time()
  brightness_gpu[(grid_size_x, grid_size_y), block_size](devSrc, devDst, threshold, brightness)
  time_gpu = time.time() - start_time_gpu
  times_gpu.append(time_gpu)
  print(f"Block size:({block_size[0]},{block_size[1]}), Time processing on GPU: {time_gpu}s")

  hostDst = devDst.copy_to_host()
  grayscale_gpu_image = np.array(hostDst)
  plt.imsave(f'labwork6b-brightness_increasing_({block_size[0]},{block_size[1]}).jpg', grayscale_gpu_image, cmap='gray')
  
  plt.title(block_size)
  plt.imshow(grayscale_gpu_image)
  plt.show()

# Implement decreasing brightness
brightness = 0
for block_size in blockSizes:
  devSrc = cuda.to_device(image)
  devDst = cuda.device_array((image_height, image_width, 3), np.uint8)

  grid_size_x = (image_height + block_size[0] - 1) // block_size[0]
  grid_size_y = (image_width + block_size[1] - 1) // block_size[1]
  start_time_gpu = time.time()
  brightness_gpu[(grid_size_x, grid_size_y), block_size](devSrc, devDst, threshold, brightness)
  time_gpu = time.time() - start_time_gpu
  times_gpu.append(time_gpu)
  print(f"Block size:({block_size[0]},{block_size[1]}), Time processing on GPU: {time_gpu}s")

  hostDst = devDst.copy_to_host()
  grayscale_gpu_image = np.array(hostDst)
  plt.imsave(f'labwork6b-brightness_decreasing_({block_size[0]},{block_size[1]}).jpg', grayscale_gpu_image, cmap='gray')
  
  plt.title(block_size)
  plt.imshow(grayscale_gpu_image)
  plt.show()
\end{lstlisting}
\subsection{Result}
\begin{lstlisting}[language=Python]
Block size:(2,2), Time processing on GPU: 0.14322471618652344s
Block size:(4,4), Time processing on GPU: 0.00029587745666503906s
Block size:(8,8), Time processing on GPU: 0.000308990478515625s
Block size:(16,16), Time processing on GPU: 0.0003211498260498047s
Block size:(32,32), Time processing on GPU: 0.00028586387634277344s
Block size:(32,22), Time processing on GPU: 0.0002949237823486328s

Block size:(2,2), Time processing on GPU: 0.0003032684326171875s
Block size:(4,4), Time processing on GPU: 0.0002770423889160156s
Block size:(8,8), Time processing on GPU: 0.0002770423889160156s
Block size:(16,16), Time processing on GPU: 0.00028014183044433594s
Block size:(32,32), Time processing on GPU: 0.0003273487091064453s
Block size:(32,22), Time processing on GPU: 0.0002617835998535156s
\end{lstlisting}

\section{Labwork 6c}
\subsection{Program}
\begin{lstlisting}[language=Python]
import numba
import numpy as np
import matplotlib.pyplot as plt
from numba import cuda
from numba import jit
from matplotlib.image import imread
import time

# Load images
first_image_path = 'labwork6c_first_image.jpg'
second_image_path = 'labwork6c_second_image.jpg'
first_image = imread(first_image_path)
second_image = imread(second_image_path)
plt.subplot(131), plt.imshow(first_image), plt.title('First image')
plt.subplot(132), plt.imshow(second_image), plt.title('Second image')

# Get the image dimensions
image_height = first_image.shape[0]
image_width = first_image.shape[1]

# Blending using GPU with 2D blocks
@cuda.jit
def blending_gpu(first_src, second_src, dst, c):
    x = cuda.threadIdx.x + cuda.blockIdx.x * cuda.blockDim.x
    y = cuda.threadIdx.y + cuda.blockIdx.y * cuda.blockDim.y
    dst[x, y, 0] = first_src[x, y, 0] * c + (1 - c) * second_src[x, y, 0]
    dst[x, y, 1] = first_src[x, y, 1] * c + (1 - c) * second_src[x, y, 1]
    dst[x, y, 2] = first_src[x, y, 2] * c + (1 - c) * second_src[x, y, 2]

# Implement with GPU
blockSizes = [(2,2),(4,4),(8, 8),(16, 16),(32,32),(32,22)]
times_gpu = []
c = 0.8
plt.show()

# Implement blending
for block_size in blockSizes:
  first_devSrc = cuda.to_device(first_image)
  second_devSrc = cuda.to_device(second_image)

  devDst = cuda.device_array((image_height, image_width, 3), np.uint8)

  grid_size_x = (image_height + block_size[0] - 1) // block_size[0]
  grid_size_y = (image_width + block_size[1] - 1) // block_size[1]
  start_time_gpu = time.time()
  blending_gpu[(grid_size_x, grid_size_y), block_size](first_devSrc, second_devSrc, devDst, c)
  time_gpu = time.time() - start_time_gpu
  times_gpu.append(time_gpu)
  print(f"Block size:({block_size[0]},{block_size[1]}), Time processing on GPU: {time_gpu}s")

  hostDst = devDst.copy_to_host()
  blending_gpu_image = np.array(hostDst)
  plt.imsave(f'labwork6c-blending_({block_size[0]},{block_size[1]}).jpg', blending_gpu_image, cmap='gray')
  
  plt.title(block_size)
  plt.imshow(blending_gpu_image)
  plt.show()
\end{lstlisting}
\subsection{Result}
\begin{lstlisting}[language=Python]
Block size:(2,2), Time processing on GPU: 0.16482853889465332s
Block size:(4,4), Time processing on GPU: 0.00038814544677734375s
Block size:(8,8), Time processing on GPU: 0.0002799034118652344s
Block size:(16,16), Time processing on GPU: 0.00036215782165527344s
Block size:(32,32), Time processing on GPU: 0.00032329559326171875s
Block size:(32,22), Time processing on GPU: 0.0003104209899902344s
\end{lstlisting}

\section{Discussion}
I have tried to implement in different block sizes in several time. 
We can see the speed will increase if we increase the block size.
The second time is always faster than the first time because it takes a lot of time for initiation.

\end{document}